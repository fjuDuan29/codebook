\documentclass[10pt,twocolumn,oneside]{article}
\setlength{\columnsep}{18pt}                    %兩欄模式的間距
\setlength{\columnseprule}{0pt}                 %兩欄模式間格線粗細

\usepackage{amsthm}                             %定義,例題
\usepackage{amssymb}
\usepackage{fontspec}                           %設定字體
\usepackage{color}
\usepackage[x11names]{xcolor}
\usepackage{listings}                           %顯示code用的
\usepackage{fancyhdr}                           %設定頁首頁尾
\usepackage{graphicx}                           %Graphic
\usepackage{enumerate}
\usepackage{titlesec}
\usepackage{amsmath}
\usepackage[CheckSingle, CJKmath]{xeCJK}
\usepackage{CJKulem}

\usepackage{amsmath, courier, listings, fancyhdr, graphicx}
\topmargin=0pt
\headsep=5pt
\textheight=740pt
\footskip=0pt
\voffset=-50pt
\textwidth=545pt
\marginparsep=0pt
\marginparwidth=0pt
\marginparpush=0pt
\oddsidemargin=0pt
\evensidemargin=0pt
\hoffset=-42pt

%\renewcommand\listfigurename{圖目錄}
%\renewcommand\listtablename{表目錄}

%%%%%%%%%%%%%%%%%%%%%%%%%%%%%

\setmainfont[
    AutoFakeSlant,
    BoldItalicFeatures={FakeSlant},
    UprightFont={* Medium},
    BoldFont={* Bold}
]{Inconsolata}
%\setmonofont{Ubuntu Mono}
\setmonofont[
    AutoFakeSlant,
    BoldItalicFeatures={FakeSlant},
    UprightFont={* Medium},
    BoldFont={* Bold}
]{Inconsolata}
\setCJKmainfont{Noto Sans CJK TC}
\XeTeXlinebreaklocale "zh"                      %中文自動換行
\XeTeXlinebreakskip = 0pt plus 1pt              %設定段落之間的距離
\setcounter{secnumdepth}{3}                     %目錄顯示第三層

%%%%%%%%%%%%%%%%%%%%%%%%%%%%%
\makeatletter
\lst@CCPutMacro\lst@ProcessOther {"2D}{\lst@ttfamily{-{}}{-{}}}
\@empty\z@\@empty
\makeatother
\lstset{                                        % Code顯示
    language=C++,                               % the language of the code
    basicstyle=\footnotesize\ttfamily,          % the size of the fonts that are used for the code
    numbers=left,                               % where to put the line-numbers
    numberstyle=\scriptsize,                    % the size of the fonts that are used for the line-numbers
    stepnumber=1,                               % the step between two line-numbers. If it's 1, each line  will be numbered
    numbersep=5pt,                              % how far the line-numbers are from the code
    backgroundcolor=\color{white},              % choose the background color. You must add \usepackage{color}
    showspaces=false,                           % show spaces adding particular underscores
    showstringspaces=false,                     % underline spaces within strings
    showtabs=false,                             % show tabs within strings adding particular underscores
    frame=false,                                % adds a frame around the code
    tabsize=2,                                  % sets default tabsize to 2 spaces
    captionpos=b,                               % sets the caption-position to bottom
    breaklines=true,                            % sets automatic line breaking
    breakatwhitespace=true,                     % sets if automatic breaks should only happen at whitespace
    escapeinside={\%*}{*)},                     % if you want to add a comment within your code
    morekeywords={*},                           % if you want to add more keywords to the set
    keywordstyle=\bfseries\color{Blue1},
    commentstyle=\itshape\color{Red1},
    stringstyle=\itshape\color{Green4},
}


\begin{document}
\pagestyle{fancy}
\fancyfoot{}
%\fancyfoot[R]{\includegraphics[width=20pt]{ironwood.jpg}}
\fancyhead[C]{team053}
\fancyhead[L]{fju_O4O_codeBook}
\fancyhead[R]{\thepage}
\renewcommand{\headrulewidth}{0.4pt}
\renewcommand{\contentsname}{Contents}

\scriptsize
\tableofcontents
\section{Section1}
    \subsection{basic}
        \lstinputlisting{Contents/section1/basic.cpp}
    \subsection{字元}
        \begin{itemize}
    \item #include<ctype.h>
    
    \item 讀取字元,可用 getchar()
    \item x = getchar();
    \item 列印字元,可用 putchar()
    \item putchar(x);
    \item putchar('\n');
    
    \item int isalnum(int c) 檢查傳遞的字符是否是字母數字。
    \item int isalpha(int c) 是否傳遞的字符是字母。
    \item int iscntrl(int c) 是否傳遞的字符是控制字符。
    \item int isdigit(int c) 是否傳遞的字符是十進制數字。
    \item int islower(int c) 檢查傳遞的字符是否是小寫字母。
    \item int isprint(int c) 檢查傳遞的字符是否是可打印的。
    \item int ispunct(int c) 檢查傳遞的字符是否是標點符號。
    \item int isspace(int c) 檢查傳遞的字符是否是空白。
    \item int isupper(int c) 檢查傳遞的字符是否是大寫字母。
    \item int isxdigit(int c) 檢查傳遞的字符是否是十六進製數字。
    \end{itemize}
    \subsection{字串}
        \begin{itemize}
    \item #include<string.h>
    
    \item strcpy	將字串 s2 拷貝到 s1	char *strcpy(char *s1, const char *s2);
    \item strncpy	將字串 s2 最多 n 個字元拷貝到 s1	char *strncpy(char *s1, const char *s2, size_t n);
    \item strcat	將字串 s2 接到 s1 的尾端	char *strcat(char *s1, const char *s2);
    \item strncat	將字串 s2 最多 n 個字元接到 s1 的尾端	char *strncat(char *s1, const char *s2, size_t);
    \item strcmp	比較 s1 與 s2 兩個字串是否相等	int strcmp(const char *s1, const char *s2);
    \item strncmp	比較 s1 與 s2 兩個字串前 n 個字元是否相等	int strncmp(const char *s1, const char *s2, size_t n);
    \item strcspn	計算經過幾個字元會在字串 s1 中遇到屬於 s2 中的字元	size_t strcspn(const char *s1, const char *s2);
    \item strspn	計算經過幾個字元會在字串 s1 中遇到不屬於 s2 中的字元	size_t strspn(const char *s1, const char *s2);
    \item strpbrk	回傳在字串 s2 中的任何字元在 s1 第一次出現位置的指標	char *strpbrk(const char *s1, const char *s2);
    \item strchr	回傳在字串 s 中,字元 c 第一次出現位置的指標	char *strchr(const char *s, int c);
    \item strrchr	回傳在字串 s 中,字元 c 最後一次出現位置的指標	char *strrchr(const char *s, int c);
    \item strstr	回傳在字串 s2 在 s1 第一次出現位置的指標	char *strstr(const char *s1, const char *s2);
    \item strtok	以字串 s2 的內容切割 s1	char *strtok(char *s1, const char *s2);
    \item strlen	計算字串的長度	size_t strlen(const char *s);
    
    \item memcpy	從 s2 所指向的資料複製 n 個字元到 s1	void *memcpy(void *s1, const void *s2, size_t n);
    \item memmove	從 s2 所指向的資料複製 n 個字元到 s1	void *memmove(void *s1, const void *s2, size_t n);
    \item memcmp	比較 s1 與 s2 前 n 個字元的資料	int memcmp(const void *s1, const void *s2, size_t n);
    \item memchr	找出字元 c 在 s 前 n 個字元第一次出現的位置	void *memchr(const void *s, int c, size_t n);
    \item memset	將 s 中前 n 個字元全部設定為 c	void *memset(void *s, int c, size_t n);
    \end{itemize}
    \subsection{型別範圍}
        \begin{itemize}
    \item int                 -2,147,483,648 ~ 2,147,483,647
    \item unsigned int	    0 ~ 4,294,967,295
    \item char                -128 ~ 127
    \item unsigned char       0 ~ 255
    \item long long           -9,223,372,036,854,775,808 ~ 9,223,372,036,854,775,807
    \item unsigned long long  0 ~ 18,446,744,073,709,551,615
    \item float               3.4E +/- 38 (7 位數)
    \item double              1.7E +/- 308 (15 位數)
    \end{itemize}
    \subsection{格式控制字串}
        \begin{itemize}
    \item printf()
    \item %ld 長整數
    \item %lld long long整數
    \item %Lf long double
    \item %o 無號八進位整數
    \item %u 無號十進位整數
    \item %x 無號十六進位整數
    \item scanf()
    \item %lf 被精度浮點數
    \item %Lf long double
    \end{itemize}
    \subsection{math}
        \begin{itemize}
    \item #include<math.h>
    
    \item double acos(double x) 返回x的反餘弦弧度。
    \item double asin(double x) 返回x的正弦弧線弧度。
    \item double atan(double x) 返回x的反正切值,以弧度為單位。
    \item double atan2(doubly y, double x) 返回y / x的以弧度為單位的反正切值,根據這兩個值,以確定正確的象限上的標誌。
    \item double cos(double x) 返回的弧度角x的餘弦值。
    \item double cosh(double x) 返回x的雙曲餘弦。
    \item double sin(double x) 返回一個弧度角x的正弦。
    \item double sinh(double x) 返回x的雙曲正弦。
    \item double tanh(double x) 返回x的雙曲正切。
    \item double exp(double x) 返回e值的第x次冪。
    \item double frexp(double x, int *exponent)
    \item The returned value is the mantissa and the integer yiibaied to by exponent is the exponent. The resultant value is x = mantissa * 2 ^ exponent.
    \item double ldexp(double x, int exponent)
    \item Returns x multiplied by 2 raised to the power of exponent.
    \item double log(double x) 返回自然對數的x(基準-E對數)。
    \item double log10(double x) 返回x的常用對數(以10為底)。
    \item double modf(double x, double *integer) 返回的值是小數成分(小數點後的部分),並設置整數的整數部分。
    \item double pow(double x, double y) 返回x的y次方。
    \item double sqrt(double x) 返回x的平方根。
    \item double ceil(double x) 返回大於或等於x的最小整數值。
    \item double fabs(double x) 返回x的絕對值
    \item double floor(double x) 返回的最大整數值小於或等於x。
    \item double fmod(double x, double y) 返回的x除以y的餘數。
    
    \item #include<stdlib.h>
    
    \item int abs(int x) 返回x的絕對值。
    \item div_t div(int numer, int denom) 數(分子numer)除以分母(分母denom)。
    \item long int labs(long int x) 返回x的絕對值。
    \item ldiv_t ldiv(long int numer, long int denom) 數(分母denom)除以分母(分子numer)。
    \item int rand(void) 返回一個取值範圍為0到RAND_MAX之間的偽隨機數。
    \item void srand(unsigned int seed) 這個函數使用rand函數隨機數生成器的種子。
    \end{itemize}

\section{Section2}
    \subsection{thm}
        \input{Contents/section2/thm.tex}

\end{document}
